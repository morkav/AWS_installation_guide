%%%%%%%%%%%%%%%%%%%%%%%%%%%%%%%%%%%%%%%
%%                                   %%        
%%  Author : Pierre-Victor Chaumier  %%
%%                                   %%
%%  Version : DEC/2014 - V0.1        %%   
%%                                   %%
%%%%%%%%%%%%%%%%%%%%%%%%%%%%%%%%%%%%%%%

\documentclass[a4paper, notitlepage, oneside, 12pt]{article}

%%%%%%%%%%%%
% Packages %
%%%%%%%%%%%%

\usepackage[latin1]{inputenc} % encodage
\usepackage[T1]{fontenc} 
\usepackage[english]{babel} % Pour avoir le support de la langue francaise
\usepackage[margin=2.5cm]{geometry} % pour modifier les marges
\usepackage{color} % pour mettre du texte en couleur
\usepackage{hyperref} % pour mettre des liens
\usepackage{changepage} % pour mettre de l'indentation sur un paragraphe entier
% \usepackage{parskip} % pour mettre des espaces entre les paragraphes
\usepackage{graphicx} % pour les images
\usepackage{tabularx} % gestion des tableaux avec taille des colonnes automatique

\setcounter{tocdepth}{4} % d�fini la hauteur de l'arborescence � faire apparaitre dans la table of content
\setlength{\parskip}{\baselineskip}
\let\EndItemize\enditemize % rajoute une ligne apr�s les listes
\def\enditemize{\EndItemize\bigskip}

%%%%%%%%%%%%
% Document %
%%%%%%%%%%%%

\begin{document}
\title{DOCUMENTATION AWS-DJANGO}
\author{Pierre-Victor \textsc{Chaumier}}
\date{D�cembre 2014}
\maketitle

\begin{enumerate}
\item launch instance

An EC2 instance is a virtual machine running in the AWS cloud. So basically, we are only going to create an ubuntu virtual machine (or anything else available) and generate a key to connect with ssh to the instance. It is on this instance that we will put our django code.

\begin{itemize}
\item go on EC2 page
\item on left panel, click on instance
\item click on launch instance button
\item select distro
\item select instance type
\item choose a security key or create a new one (DO NOT LOOSE IT, it is used to connect to the instance. Key lost means having to kill the instance and start again..) 
I put it in the ~/.ssh/ folder
\end{itemize}
Now your instance should be up running


\item Create a DB

Here, we are going to launch a virtual machine optimized to run databases. I would recommend Postgresql because I like it (for no specific reason, never had any problem with MySQL). Then we are going to set the security rules so that only our EC2 instance can access the database.

\begin{itemize}
\item on the main AWS panel, go on RDS
\item click on get started now
\item Select database
\item production (be careful dafault one is not free)
\item choose details and settings (for instance my DB instance identifier is "dbinstance", my master username is "dbadmin" and my password is "password" no kidding ^^)
\item complete advanced settings. BE CAREFUL with publicly accessible. I put no so that only my instances on AWS can access it. Choose a dbname (mine is dbname\_of\_project)
\item click on launching db instance. Creating a DB can take a little time so be patient.
\item create a security group (they are available from the ec2 dashboard on the left) name it as you wish (i choose dbsecuritygroup), inbound should be HTTP with custom IP for the source. The custom IP will be the identifier of the security group of your instance sg... 
\end{itemize}
That is it, your DB should be safe and running now.

\item Create a S3 instance to store media files

S3 instance are specially designed to store files. I have not finished this part of the tutorial so you will have to store all the files on you EC2 instance for now. If you want to store it on a S3 instance, check the caktus group website. They did a good job explaining how to serve static and media files on AWS.


\item Setting up the shortcut to connect with ssh using config file in .ssh/ folder

This part explains only how to create and configure a ssh with a local key.

\begin{itemize}
\item go in your ~/.ssh/ folder
\item create a file named config
\item fill it as follow :
\begin{verbatim}
Host name_of_shortcut
HostName addresse of your instance ec2...
User name_of_user (ubuntu for ubuntu instances)
IdentityFile /path/to/key/file/ (generated earlier)
\end{verbatim}
\item change rights 
\begin{verbatim}chmod 700 for ~/.ssh/ folder\end{verbatim}
\begin{verbatim}chmod 600 for the key\end{verbatim}
\item run commande : \begin{verbatim}ssh name\_of\_shortcut\end{verbatim} to connect to the instance
\end{itemize}
Now you should be connected to the terminal of the instance.

\item installation (postgre etc..)
simply run the following commands (for debian/ubuntu instance):
\begin{itemize}
\item \begin{verbatim}sudo apt-get update\end{verbatim}
\item \begin{verbatim}sudo apt-get upgrade\end{verbatim}
\item \begin{verbatim}sudo apt-get install -y python git python-pip postgresql 
  postgresql-server-dev-all python-dev libpq-dev supervisor nginx vim 
  curl ntp libncurses5-dev make build-essential libssl-dev zlib1g-dev 
  libbz2-dev libreadline-dev libsqlite3-dev wget curl llvm tcl8.5\end{verbatim}
\end{itemize}

\item clone your repository

This part is up to you. Git is installed so if you use git+github then you just have to git clone your repo and that is it. The point here is to put your django project on the instance.

\item setting git options 

tape all the following commands to set the options:
\begin{verbatim}
git config --global color.diff auto
git config --global color.status auto
git config --global color.branch auto
git config --global alias.br branch
git config --global alias.ci commit
git config --global alias.co checkout
git config --global alias.st status
git config --global user.name "votre_pseudo"
git config --global user.email moi@email.com
\end{verbatim}

\item installation pyenv, pyenv virtualenv, pip and requirements.txt (if you have one)

Normally, git should be installed by now. It is time to install pyenv and pyenv wrapper. \href{http://fgimian.github.io/blog/2014/04/20/better-python-version-and-environment-management-with-pyenv/}{Here} is a good start. However, do not install following what is in the link but what is above.
\begin{itemize}
\item To install pyenv, follow the very well made guide here : \url{https://github.com/yyuu/pyenv\#installation} (source the .bash\_profile file after modification or it might not be taken into account)
\item now install virtualenv following instructions here : \url{https://github.com/yyuu/pyenv-virtualenv}
\item create a virtualenv with this command : pyenv virtualenv 2.7.8 nom\_du\_virtualenv 
\item to install pip ... well do nothing, the previous command took care of it :D
\item use pip install -r /path/to/requirement.txt to install all the required program with pip (fairly practical..)
\item to activate the virtual env do : pyenv activate name\_virtualenv

\end{itemize}

\item configuration of supervisor
It is a program that launch and supervise different process automatically for us (thx dude). Here is how to configure it :
\begin{itemize}
\item sudo vim /etc/supervisor/conf.d/django.conf
\item fill the document with :
\begin{verbatim}
[program:name_wsgi_process]
command=/home/ubuntu/.pyenv/versions/name_project_env/bin/gunicorn 
    name_project.wsgi:application -b 127.0.0.1:8005 -w 4 -t 30
environment=LANG="en_US.UTF-8"
directory=/home/ubuntu/name_project/name_project/
user=name_of_user (ubuntu in general)
autostart=True
autorestart=True
stdout_logfile=/var/log/django.log
stderr_logfile=/var/log/django.log
\end{verbatim}
\item enter : sudo supervisorctl update. This should returns added process group.
\item for more info on the different part, check this link \url{http://reustle.io/blog/managing-long-running-processes-with-supervisor}
\item to check that everything went alright, type this : \\
gunicorn --check-config name\_wsgi\_process
\item after modifications of django.conf file, you can launch these command to make them taken into account.\\
supervisorctl reread\\
supervisorctl reload\\
\end{itemize}

\item configure nginx
\begin{itemize}
\item run sudo vim /etc/nginx/sites-available/django
\item put this in it replacing what should be replaced.
\begin{verbatim}
server_names_hash_bucket_size 350;

underscores_in_headers on;

server {
    listen 80;
    server_name project_name;
    client_max_body_size 10m;
    set $static_root "/path/to/static/files";
    set $media_root "/path/to/media/files";

    location /static {
        alias $static_root;
        expires max;
    }

    location /media {
        alias $media_root;
        expires max;
    }
 
     location = /favicon.ico {
        rewrite (.*) /static/ico/favicon.ico;
     }
 
     location / {
        proxy_set_header  X-Real-IP  $remote_addr;
        proxy_set_header  X-Forwarded-For $proxy_add_x_forwarded_for;
        proxy_set_header Host $http_host;
        proxy_pass http://127.0.0.1:8005;
        proxy_read_timeout 120;
        proxy_connect_timeout 120;
    }
}
\end{verbatim}
\item run \begin{verbatim}cd /etc/nginx/sites-enabled/\end{verbatim}
\item if there is a default file in it, remove it (sudo rm default)
\item creation of a symbolic link to our file : \begin{verbatim}sudo ln -s /etc/nginx/sites-available/django django\end{verbatim}
\item to launch the server type : \begin{verbatim}sudo service nginx start\end{verbatim}
\end{itemize}

\item every time you modify something, the dev server relaunch by itself, here it is not the case. You have to type : 
\begin{verbatim}
sudo supervisorctl restart name_wsgi_process
sudo service nginx reload
\end{verbatim}
\end{enumerate}

\end{document}